
\documentclass[12pt]{article} 
\usepackage[utf8]{inputenc}
\usepackage[slovak]{babel}
\usepackage[hidelinks,unicode = true]{hyperref}
\usepackage{outline}
\usepackage{graphicx}
\usepackage{longtable} %pro tabulky delší než jedna stránka
%\usepackage{biblatex}
%\addbibresource{literatura.bib}
\usepackage{cite}
\usepackage{caption}
%\usepackage{float} %upevneni tabulky
%\restylefloat{table}
\setcounter{secnumdepth}{3}
\setcounter{tocdepth}{3}

%===========================================================================
\begin{document}           % Konec preambule a zároveň začátek vlastního textu
\begin{titlepage}
\centering
\Large \textbf{České vysoké učení technické v Praze }\\ Fakulta stavební
\vspace{2cm}

\begin{figure}[h!] %logoCVUT
\centering
\includegraphics[width=7cm]{./img/cvut.png}
\end{figure}
 
\Large \textbf{155ADKG Algoritmy v digitální kartografii}
\vspace{1cm}

\LARGE  \textbf{Množinové operace s polygony}
\vspace{3cm}

\Large Bc. Lukáš Kettner Bc. Martin Hulín \\ 17.12.2019

 \thispagestyle{empty} %neočísluje první stránku
\end{titlepage}

\tableofcontents    % vytváří  Obsah 
\newpage %začne na nové stránce
%------------------------------------------------------------------------
\section{Zadanie}

\begin{figure}[h!]
	\includegraphics[clip, trim=0cm 4cm 0cm 3cm, width=1.2\textwidth]{zadani.pdf}
\end{figure}
\subsection{Bonusové úlohy}
V rámci úlohy sú vypracované tieto bonusové úlohy

\begin{itemize}
\item Riešenie pre polygóny obsahujúce otvory
\end{itemize}

%------------------------------------------------------------------------
\clearpage 
\section{Popis a rozbor problému}
Podstatou úlohy je tvorba aplikácie, v ktorej grafickom rozhraní bude možné prevádzať základné množinové operácie. V rámci úlohy sa zoberáme operáciami prienik, zjednotenie a rozdiel polygónov A a B.

\begin{center}
   \includegraphics[width=12cm]{./img/operac.png}
   \captionof{figure}{Typy množinových operácií s polygónmi}
\end{center}
%------------------------------------------------------------------------
\clearpage 
\section {Popis použitých algoritmov}
\subsection {Výpočet priesečníkom, zotriedenie a aktualizácia}
Využili sme funkciu get2LinesPosition. Táto funkcia kontroluje hrany z polygonu A a polygonu B na existenciu priesečníku. V úlohe bol použitý datový typ QPointFB, ktorý uchováva hodnoty parametrov alfa a beta. Tento typ je odvodený od typu QPointF.  Pokiaľ priesečník existoval spočítali sme jeho súradnice. 

Pri výpočte priesečníku môžu nastať nasledujúce možnosti :
\begin{enumerate}
\item úsečky sú kolineárne
\item úsečky sú rovnobežné
\item úsečky sú rôznobežné
\item úsečky sú mimobežné
\end{enumerate}

Tieto hodnoty sa ukladajúdo datového typu map - kľúč je parameter alfa / beta,  hodnota priesečník. Priesečníky boli ďalej vložené do správneho polygónu na správnu pozíciu pomocou funkcie processIntersection.

\subsubsection {Implementácia metódy processIntersection}
\begin{enumerate}
\item Nastavenie tolerancie epsilon
\item $ if ((t >= epsilon ) \&\& (t <= 1- epsilon ) $
\item $ i += 1$
\item $polygon.insert(polygon.begin() + i,pi $ // priraď priesečník polygónu na pozíciu
\end{enumerate}

\subsubsection {Implementácia funkcie computePolygonIntersection}
\begin{enumerate}
\item Cyklus $ for ( int i = 0 ; i < pa.size(); i++)$ - prechádame celý polygón A
\item Vytvotenie map $ std::map< double, QPointFB> intersections$
\item  Cyklus $ for ( int j = 0 ; j < pb.size(); j++)$ - prechádame celý polygón B
\item \hspace {1.0cm}   $ if (get2LinesPosition(...) == INTERSECTED $ podmienka ak existuje prisečník
\item \hspace {1.5cm}  Získaj hodnoty alpha, beta, ulož priesečník do mapy na základe alpha $intersections[alpha]=p_i $
\item \hspace {1.5cm} $processIntersection(pi, beta, pb, j) $
\item \hspace {1.0cm} Ak bol nájdený aspoň jeden priesečník 
\item \hspace {1.0cm} prejdi mapu $for (std::pair<double, QPointFB> item:intersections)$
\item \hspace {1.0cm}získaj druhú hodnotu z páru  $QPointFB pi = item.second$
\item \hspace {1.0cm} $processIntersection(pi, alfa, pa, i) $
\end{enumerate}

\subsection {Ohodnotenie vrcholov}
Tento algoritmus uplatňuje ako ohodnocovacie pravidlo polohu vrcholu v polygóne voči druhému vrcholu. Rozsah hodnotenia v závislosti na polohe môže byť Inner, Outer, On. Tieto hodnoty boli uložené do nového datového typu TPointPolygonPosition. K určeniu polohy sme využili Winding Number algoritmus.

\subsubsection{Implementácia metódy setPosition}
\begin{enumerate}
\item Cyklus $ for(int i = 0; i < n; i++)$ - prechádame celý polygón A
\item Výpočet stredového bodu hrany 
\item \hspace {1.0cm} $double mx = (pa[i].x() + pa[(i + 1)\%n].x()) / 2;$
\item \hspace {1.0cm} $double my = (pa[i].y() + pa[(i + 1)\%n].y()) / 2;$
\item  Uloženie bodu $ QPointFB m(mx, my);$
\item Určenie polohy metodou Winding Number
\newline \hspace {1.0cm}   $TPointPolygonPosition position = positionPointPolygonWinding(m, pb);$
\item \hspace {1.0cm} Uloženie pozície počiatočného vrholu hrany
\end{enumerate}

\subsection {Ohodnotenie hran}
Výber hran pre množinové operácie znázorňuje nasledujúca tabuľka.

 Operácia \hspace {1.5cm} PolygonA \hspace {1.0cm} PolygonB 
\newline Union \hspace {1.5cm} outer \hspace {1.0cm} outer 
\newline Intersect \hspace {1.4cm} inner \hspace {1.0cm} inner 
\newline DifferenceAB \hspace {1.0cm} outer \hspace {1.0cm} inner 
\newline DifferenceBA \hspace {1.0cm} inner \hspace {1.0cm}outer

\subsection {Vytvorenie hran}
Vrcholy, ktorým náleží príslušne ohodnotenie sme následne spojili do hrán a uložili do vektoru, ktorý je vykreslovaný.

\subsubsection{Implementácia metódy selectEdges}
\begin{enumerate}
\item Cyklus $ for(int i = 0; i < n; i++)$  prechádzame celý polygon
\item Nájdenie vhodnej hrany
\item \hspace {1.0cm} $Edge e (pol[i], pol[(i+1)\%pol.size()]);$ vytvorenie hrany
\item \hspace {1.0cm} $edges.push_back(e);$ pridanie hrany do vektoru hran
\end{enumerate}
\clearpage 
%------------------------------------------------------------------------
\section{Vstupné dáta}
Vstupnými dátami sú dva polygóny, naklikané ručne v grafickom rozhraní aplikácie. Pomocou tlačítka PolagonA/B je možné prepínať medzi kresbou jednotlivých polygónov.

\section{Výstupné - generovanie množinových operácií}
Výstupnými dátami je graficky reprezentované zobrazenie množinových operácií.

%------------------------------------------------------------------------

\section{Ukážka vytvorenej aplikácie}

\begin{center}
   \includegraphics[width=10cm]{./img/aplikacia1.png}
   \captionof{figure}{Ukážka grafického rozhrania aplikácie}
\end{center}

\begin{center}
   \includegraphics[width=10cm]{./img/aplikacia2.png}
   \captionof{figure}{Ukážka grafického rozhrania aplikácie - 2 polygóny}
\end{center}

\begin{center}
   \includegraphics[width=10cm]{./img/aplikacia_union.png}
   \captionof{figure}{Ukážka grafického rozhrania aplikácie - Union}
\end{center}

\begin{center}
   \includegraphics[width=10cm]{./img/aplikacia_intersect.png}
   \captionof{figure}{Ukážka grafického rozhrania aplikácie - Intersect}
\end{center}

\begin{center}
   \includegraphics[width=10cm]{./img/aplikacia_AB.png}
   \captionof{figure}{Ukážka grafického rozhrania aplikácie - Difference AB}
\end{center}

\begin{center}
   \includegraphics[width=10cm]{./img/aplikacia_BA.png}
   \captionof{figure}{Ukážka grafického rozhrania aplikácie - Difference BA}
\end{center}

\begin{center}
   \includegraphics[width=10cm]{./img/aplikacia_diera_Union.png}
   \captionof{figure}{Ukážka grafického rozhrania aplikácie - Holes pri operácii Union}
\end{center}

%-------------------------------------------------------------------------
\section{Dokumentácia}
\subsection{Trieda Algorithms}
Triedu Algorithms sme použili pre deklarovanie funkcií pre výpočtové algoritmy tvorby množinových operácií s polygónmi.

\subsubsection{Metódy}

\begin{enumerate}
\item[] \underline{getAngle2Vectors}
\begin{itemize}
\item Slúži k určeniu uhlu medzi dvoma priamkami. Jej návratovou hodnotou je double
\item na vstupe má : súradnice bodov $p_1, p_2, p_3, p_4$ určujúcich prvú a druhú priamku
\item výstupom je hodnota uhlu medzi priamkami 
\end{itemize}

\item[] \underline{getPointLinePosition}
\begin{itemize}
\item Slúži na určenie polohy bodu voči priamke. Jej návratovou hodnotou je integer.
\item na vstupe má : súradnice určovaného bodu q , súradnice bodov priamky $p_1$ $p_2$
\item na výstupe hodnoty :
\item[] - LeftHp
\item[] - RightHp
\item[] - Colinear
\end{itemize}

\item[] \underline{positionPointPolygonWinding}
\begin{itemize}
\item slúži k určeniu polohy bodu prostredníctvom Winding Number algoritmu. Jej návratový typ je integer.
\item na vstupe má : QPointFB q – bod ktorého polohu určujeme, std$::$vector$<$QPointFB$>$ pol – polygon, voči ktorému určujeme polohu bodu q
\item výstupom je hodnota :
\item[] - Outer
\item[] - Inner
\end{itemize}

\item[] \underline{get2LinesPosition}
\begin{itemize}
\item funkcia slúžia k výpočtu polohy dvoch priavok voči sebe. 
\item na vstupe je sú body QPointFB p1, p2, p3, p4, pi
\item na výstupe hodnoty :
\item[] - Identical
\item[] - Paralel
\item[] - Intersected
\item[] - NonIntersected
\end{itemize}


\item[] \underline{booleanOperations}
\begin{itemize}
\item funkcia slúžia k prevedeniu množinových operácií.. 
\item na vstupe je sú polygóny bodov QPointFB polygonA, polygonB a typ operácie 
\item na výstupe je vektor hrán odpovedajúci zvolenej operácii pre polygóny
\end{itemize}


\item[] \underline{processIntersection}
\begin{itemize}
\item funkcia slúžia k prevedeniu zaradenia vypočítaného priesečníku na správnu pozíciu v príslušnom polygóne. Jej návratovym typom je void. 
\end{itemize}


\item[] \underline{computePolygonIntersection}
\begin{itemize}
\item funkcia slúžia k výpočtu priesečníku. Jej návratovym typom je void. 
\end{itemize}


\item[] \underline{setPositionsAB}
\begin{itemize}
\item funkcia je pomocnou funkciou pre boolenOperations. Jej návratovym typom je void. 
\end{itemize}


\item[] \underline{setPositions}
\begin{itemize}
\item funkcia slúži k určeniu pozície hrany. Jej návratovym typom je void. 
\end{itemize}


\item[] \underline{selectEdges}
\begin{itemize}
\item funkcia slúži k vybraniu príslušných hrán. Jej návratovym typom je void. 
\end{itemize}


\item[] \underline{booleanOperationsHoles}
\begin{itemize}
\item funkcia slúžia k prevedeniu množinových operácií pri zadaní Holes. 
\item na vstupe je sú polygóny bodov QPointFB polygonA, polygonB a typ operácie 
\item na výstupe je vektor hrán odpovedajúci zvolenej operácii pre polygóny
\end{itemize}


\item[] \underline{mergeVectors}
\begin{itemize}
\item funkcia slúži k zlúčeniu vektorov hrán. Jej návratovym typom je void. 
\end{itemize}
\end{enumerate}

\subsection{Trieda Draw}
Trieda Draw slúži ku grafickému vykresleniu množiny modov a konvexnej obálky nad touto množinou.

\subsubsection{Členské premenné}
\begin{enumerate}

\item[] \underline {std::vector}$<${QPoint}$>${points}
\begin{itemize}
\item vektor bodov okolo ktorých vytvárame konvexnú obálku
\end{itemize}
\item[] \underline {QPolygon ch}
\begin{itemize}
\item polygón obsahujúci body konvexnej obálky
\end{itemize}
\end{enumerate}

\subsubsection{Metódy}
\begin{enumerate}
\item[] \underline{paintEvent}
\begin{itemize}
\item slúži k vykresleniu naklikaných a vygenerovaných bodov, vykresleniu konvexnej obálky. Návratovým typom je void.
\end{itemize}
\item[] \underline{void mousePressEvent}
\begin{itemize}
\item slúži k vykresleniu bodu  stlačením tlačidla myši, v okamihu stlačenia tlačidla na myši sa uložia súradnice bodu do vektoru points. Návratovým typom je void.
\end{itemize}
\item[] \underline{void clearCH}
\begin{itemize}
\item slúži k vymazaniu konvexnej obálky. Návratovým typom je void.
\end{itemize}
\item[] \underline{void clearPoints}
\begin{itemize}
\item slúži k vymazaniu množiny bodov. Návratovým typom je void.
\end{itemize}
\item[] \underline {std::vector}$<${QPoint}$>${getPoints}
\begin{itemize}
\item vektor, ktorý slúži k vráteniu množiny bodov points.
\end{itemize}
\item[] \underline {setCH}
\begin{itemize}
\item slúži na prevedenie konvexnej obálky do vykresľovacieho okna.
\end{itemize}
\item[] \underline {generatePoints}
\begin{itemize}
\item slúži ku generovaniu množiny bodov. Na vstupe je zadaná metóda, počet bodov, šírka a výška.
\item na výstupe je vygenerovaná množina bodov podľa užívateľského zadania.
\end{itemize}
\item[] \underline {std::vector}$<${QPoint}$>${generatePointsU2}
\begin{itemize}
\item slúži ku generovaniu množiny bodov. Na vstupe je zadaná metóda, počet bodov, šírka a výška.
\item na výstupe je vygenerovaná množina bodov podľa užívateľského zadania. Generovanie konvexnej obálky prebehne automaticky 10x pre všetky typy tvaru generovanej množiny bodov a počty generovaných bodov v intervale od 1 000 do 1 000 000.
\end{itemize}
\end{enumerate}

\subsection{Tridy SortByX, SortByY}
Sú to triedy, ktoré obsahujú zoraďovacie kritériá. Pomocou týchto funkcií zoradíme súbor bodov podľa X alebo podľa Y súradnice.

\subsection{Trieda Widget}
Tieda Widget obashuje metódy ktoré sú odkazom na sloty umožňujúce vykonávať príkazy z grafického rozhrania aplikácie. Nemajú žiadne vstupné hodnoty, návratovým typom je void.

\subsubsection{Metódy}
\begin{enumerate}
\item[] \underline{on\_pushButton\_createCH\_clicked}
\begin{itemize}
\item tlačidlo \textbf{Create convex hull} po kliknutí naň sa vygeneruje konvexná obálka množiny bodov
\end{itemize}

\item[] \underline{on\_pushButton\_clearPoints\_clicked}
\begin{itemize}
\item tlačidlo \textbf{Clear points}  po kliknutí naň sa vymaže množina bodov
\end{itemize}

\item[] \underline{on\_pushButton\_clearCH\_clicked}
\begin{itemize}
\item tlačidlo \textbf{Clear convex hull}  po kliknutí naň sa vymaže konvexná obálka
\end{itemize}

\item[] \underline{on\_pushButton\_generatePoints\_clicked}
\begin{itemize}
\item tlačidlo \textbf{Generate points}  po kliknutí naň sa vygenerujú body v zvolenom tvare a počte
\end{itemize}

\item[] \underline{on\_pushButton\_solveU2\_clicked}
\begin{itemize}
\item tlačidlo \textbf{Solve U2}  po kliknutí naň sa vykoná automatické generovanie konvexnej obálky 10x pre každý typ rozmiestnenia bodov ( raster, kruh, náhodné rozmiestnenie bodov) a počet bodov n $\in$ $\langle$ 1000, 5000, 10 000, 25 000, 50 000, 75 000, 100 000, 250 000, 500 000, 750 000, 1000 000 $\rangle$ . Pre každé generovanie konvexnej obálky je počítaná doba behu, ktorá sa spolu s počtom a typom generovaných bodov, typom algoritmu a poradím opakovania je ukladaná do textového súboru
\end{itemize}

\end{enumerate}

%-------------------------------------------------------------------------

\clearpage
\section{Záver}
Výsledkom úlohy je funkčná aplikácia a grafická prezentácia doby trvania jednotlivých algoritmov pre rôzne množiny bodov. Po vypracovaní úlohy sme došli k nasledujúcemu záveru. Najvhodnejší algoritmus pre generovanie konvexných obálok je Sweep Line. Táto metóda dosiahla najlepšie výsledky vo všetkých testovacích prípadoch. Obzvlásť výrazný časový rozdie oproti ostatným algoritmoml bol pri zväčšujúcej sa množine bodov usporiadanej v kruhovom tvare. Algoritmy Graham Scan a Jarvis Scan sú o poznanie pomalšie, hlavne pri rastri a náhodne generovanej množine bodov. Algoritmus Quick Hull dosahuje pre rastrovú a náhodnú množinu bodov takmer totožnú rýchlosť ako algoritmus Sweep Line, pri kruhovom rozložení bodov je pomalší, obdobne rýchly ako Graham Scan.

V úlohe sme sa neimplementovali generovanie množiny bodov v tvare  elipsy a star - shaped rozloženia. Bolo by zaujímavé porovnať doby behov algoritmov aj pre takéto rozloženia množín bodov.

%-------------------------------------------------------------------------

\newpage
%-------------------------------------------------------------------------

%Zobrazeni seznamu obrazku
%\cleardoublepage
%\addcontentsline{toc}{chapter}{\listfigurename}
\listoffigures

%-------------------------------------------------------------------------    
\end{document}             % Konec dokumentu.
